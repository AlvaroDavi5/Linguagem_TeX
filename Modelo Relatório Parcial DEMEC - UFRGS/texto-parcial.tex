%%%%%%%%%%%%%%%%%%%%%%%%%%%%%%%
% TEXTO DO RELATORIO PARCIAL 
%%%%%%%%%%%%%%%%%%%%%%%%%%%%%%%

%%%%%%%%%%%%%%%%%%
%%%% OBJETIVO %%%%
\section{Objetivo}

%Implementar otimização topológica baseada em elasticidade não-local e
%comparar com métodos apresentados na literatura.
\begin{quote}Corresponde a responder a pergunta: O QUE tu farás como trabalho, ou ainda QUAL o tema do teu trabalho.
\end{quote}

O objetivo deste trabalho é se auto-justificar por meio da aplicação de lógica
sinuosa e convoluta, numa tentativa de convencer a banca do mérito do trabalho.

%%%%%%%%%%%%%%%%%%%%%%%
%%%% JUSTIFICATIVA %%%%
\section{Justificativa}

\begin{quote}
Deves expor aqui a relevância do tema escolhido. Corresponde a responder a
pergunta: POR QUE vais desenvolver esse trabalho?
\end{quote}

A importância de terminar o TCC e se formar é justificada pela diferença
entre o valor de uma bolsa de estágio e um salário de engenheiro.

\begin{equation}\label{eq:mat}
E_e = E_{min} + x_e^p(E_0 - E_{min}),~~ x \in [0,1]
\end{equation}

Um pouco de matemática sempre ajuda:

\begin{equation}\label{eq:obj}
	\begin{aligned}
		\underset{\mathbf{x}}{\text{minimizar}} :& \quad c(\mathbf{x}) =
			\mathbf{u}^T \mathbf{K u} =
			\sum_{e=1}^N E_e (x_e) \mathbf{u}_e^T \mathbf{K}_0 \mathbf{u}_e\\
		\text{sujeito a} :& \quad V(\mathbf{x})/V(\mathbf{1}) = f_v\\
				  & \quad\mathbf{K u} = \mathbf{f}\\
		    & \quad\mathbf{0} \leq \mathbf{x} \leq \mathbf{1}
	\end{aligned}
\end{equation}


E mais um pouco:

\begin{equation}\label{eq:sen}
\frac{\partial c(\mathbf{x})}{\partial x_e} = p x_e^{p-1}(E_0-E_{min})
\mathbf{u}^T_e \mathbf{K}_0 \mathbf{u}_e
\end{equation}

O problema descrito pelas equações \ref{eq:mat}, \ref{eq:obj} e \autoref{eq:sen}
pode ser visto em \citeonline[cap. 1.3]{bendsoetopology}.


%%%%%%%%%%%%%%%%%%%%%
%%%% METODOLOGIA %%%%
\section{Metodologia}

\begin{quote}
Deves mostrar aqui os passos que serão feitos para a realização do trabalho.
Devem ser citados os procedimentos, tais como revisão dos assuntos relacionados,
a montagem de algum projeto, o cálculo de alguma instalação, a comparação econômica, etc.
Corresponde a responder a pergunta: COMO vai ser feito o trabalho?
\end{quote}

O trabalho pode ser dividido nos seguintes passos:

\begin{itemize}
\item Revisão da literatura referente a otimização topológica;
\item Formulação de uma proposta de método;
\item Implementação do método proposto;
\item Aplicação do método a problemas padrão;
\item Avaliação da performance do método por comparação com resultados
	analíticos encontrados na literatura e com resultados de métodos existentes.
\end{itemize}

Os passos serão feitos em ordem ou não.

%%%%%%%%%%%%%%%%%%%%%%%
%%%% ESTÁGIO ATUAL %%%%
\section{Estágio atual do desenvolvimento}\label{cap_crono}

\begin{quote}
Indicar o estado atual do trabalho, seguido de um cronograma semanal. Esse cronograma deve
incluir todas as etapas do trabalho até a sua apresentação oral.
\end{quote}

Atualmente o trabalho está na etapa onde coisas precisam ser feitas (ver cronograma na \autoref{t:crono}).

\begin{table}[hb]
	\caption{\label{t:crono}Cronograma do trabalho.}
	\center
	\includegraphics[width=.98\textwidth]{crono.pdf}
\end{table}
