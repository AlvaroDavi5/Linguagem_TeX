\documentclass[a4paper]{article}
% incusao de pacotes (nao-necessario aqui)
\usepackage{amsmath, calc, xcolor} % biblioteca de formulas matematicas
\usepackage[utf8]{inputenc} % leitura de caracteres unicode
\usepackage[english]{babel} % interpretacao do idioma pt-br


% titulo e autor
\title{Resolução - Ex02: Circuitos Resistivos em Série e Paralelo}
\author{por Álvaro Davi S. Alves
	\thanks{Estudante de Engenharia da Computação}
}


% inicio do documento
\begin{document}
	\maketitle
	Universidade Federal do Espírito Santo

	\newpage
	Intro salveee \\*
	Olá \\
	Quem é? \\

	$\sin_\Theta = f(x)$

	\begin{equation}
		\lim_{t \rightarrow \infty}
		\int_{a}^{t} f(x) dx
	\end{equation}
\end{document}

% ---------------------------------------------------------------

\ifnot
% -- classe do documento --
\documentclass[
	% -- classe de construcao --
		11pt, % tamanho da fonte
		oneside, % impressao em apenas um lado
		a4paper, % formato do papel: A4
		article, % classe do documento: Artigo
	% -- opcoes da classe abntex2 --
		%chapter=TITLE, % titulos de capitulos convertidos em letras maiusculas
		%section=TITLE, % titulos de secoes convertidos em letras maiusculas
		%subsection=TITLE, % titulos de subsecoes convertidos em letras maiusculas
	% -- opções do pacote babel --
		english, % idioma adicional para hifenizacao
		spanish, % idioma adicional para hifenizacao
		brazil % o ultimo idioma e o principal do documento
]{abntex2} % padrao ABNT2


% -- incusao de pacotes --
\usepackage[utf8]{inputenc} % codificacao do documento para leitura de caracteres unicode
\usepackage[brazil]{babel} % interpretacao do idioma pt-br
\usepackage{lmodern} % fonte Latin Modern
\usepackage{indentfirst} % indenta o primeiro paragrafo de cada secao
\usepackage{microtype} % para melhorias de justificação
\usepackage{color} % controle das cores
\usepackage[T1]{fontenc} % selecao de codigos de fonte
\usepackage{minted} % destaque de cores para codigos
\usepackage{amsmath, amsfonts, calc, xcolor} % inclusao de formulas matematicas
\usepackage{graphicx} % inclusao de graficos
\usepackage[brazilian, hyperpageref]{backref} % paginas de citacoes
\usepackage[alf, abnt-etal-list=0, abnt-etal-cite=1]{abntex2cite} % citacoes padrao ABNT
\usepackage{lastpage} % usado pela ficha catalografica


% -- configuracoes de estilos de capitulos --
\numberwithin{equation}{section}
\numberwithin{figure}{section}
\numberwithin{table}{section}
\linespread{0.95}

\renewcommand{\ABNTEXpartfontsize}{\normalsize}
\renewcommand{\ABNTEXchapterfontsize}{\normalsize}
\renewcommand{\ABNTEXsectionfontsize}{\normalsize}
\renewcommand{\ABNTEXsubsectionfontsize}{\normalsize}

\renewcommand{\familydefault}{\sfdefault}
\renewcommand{\rmdefault}{phv}
\renewcommand{\sfdefault}{phv}
\renewcommand{\ttdefault}{pcr}


% -- comando para insercao de capa --
\renewcommand{\imprimircapa}{
	\begin{capa}
		\center
		\ABNTEXchapterfont\ABNTEXchapterfontsize
		\imprimirinstituicao
		\vfill
		\imprimirtitulo
		\par
		por
		\par
		\imprimirautor
		\vfill
		RELATÓRIO PARCIAL
		\vfill
		\imprimirlocal, \imprimirdata
		\vspace*{1cm}
	\end{capa}
}


% -- informacoes da capa --
\title{Resolução - Ex02: Circuitos Resistivos em Série e Paralelo}
\author{por Álvaro Davi S. Alves
	\thanks{Estudante de Engenharia da Computação}
}
\local{Goiabeiras - Vitória - ES}
\data{março de 2015}
\instituicao{
	Universidade Federal do Espírito Santo
	\par
	Curso de Engenharia da Computação
}


% -- configuracoes do PDF --
\makeatletter
	\hypersetup{
		%pagebackref=true,
		pdftitle={\@title}, 
		pdfauthor={\@author},
		pdfcreator={LaTeX with abnTeX2},
		pdfkeywords={abnt}{latex}{abntex}{abntex2}{trabalho acadêmico}, 
		colorlinks=true, % links coloridos
		linkcolor=blue,          	% color of internal links
		citecolor=blue,        		% color of links to bibliography
		filecolor=magenta,
		urlcolor=blue,
		bookmarksdepth=4
	}
\makeatother


% -- compilar o indice --
\makeindex


% -- inicio do documento --
\begin{document}
	\frenchspacing % retirar espaço extra obsoleto entre as frases

	\imprimircapa
	%\maketitle % criar o titulo
	\phantompart % adicionar espaco para sumario

	\newpage
	Intro salveee \\*
	Olá \\

	$\sin_\Theta = f(x)$

	\begin{equation}
		\lim_{t \rightarrow \infty}
		\int_{a}^{t} f(x) dx
	\end{equation}

	Quem é? \\

	\bibliography{referencias}
\end{document}
\fi
