% parcial.tex
% ------------------------------------------------------------------------
% ------------------------------------------------------------------------
% UFRGS - DEMEC : Modelo de Relatório Parcial
% Copyright 2015 Ricardo Leuck
%
% This work may be distributed and/or modified under the
% conditions of the LaTeX Project Public License, either version 1.3
% of this license or (at your option) any later version.
% The latest version of this license is in
%   http://www.latex-project.org/lppl.txt
% and version 1.3 or later is part of all distributions of LaTeX
% version 2005/12/01 or later.
%
% This work has the LPPL maintenance status `maintained'.
% 
% The Current Maintainer of this work is Ricardo Leuck.
%
% This work consists of the files parcial.tex, texto-parcial.tex,
% crono.pdf and referencias.bib.
% ------------------------------------------------------------------------
% ------------------------------------------------------------------------

\documentclass[
	% -- opções da classe memoir --
	11pt,				% tamanho da fonte
	%openright,			% capítulos começam em pág ímpar (insere página vazia caso preciso)
	oneside,			% para impressão em verso e anverso. Oposto a oneside
	a4paper,			% tamanho do papel. 
    article,
	% -- opções da classe abntex2 --
	chapter=TITLE,		% títulos de capítulos convertidos em letras maiúsculas
	section=TITLE,		% títulos de seções convertidos em letras maiúsculas
	%subsection=TITLE,	% títulos de subseções convertidos em letras maiúsculas
	%subsubsection=TITLE,% títulos de subsubseções convertidos em letras maiúsculas
	% -- opções do pacote babel --
	english,			% idioma adicional para hifenização
	%french,				% idioma adicional para hifenização
	%spanish,			% idioma adicional para hifenização
	brazil				% o último idioma é o principal do documento
	]{abntex2}

% ---
% Pacotes básicos 
% ---
\usepackage{lmodern}			% Usa a fonte Latin Modern			
\usepackage[T1]{fontenc}		% Selecao de codigos de fonte.
\usepackage[utf8]{inputenc}		% Codificacao do documento (conversão automática dos acentos)
\usepackage{lastpage}			% Usado pela Ficha catalográfica
\usepackage{indentfirst}		% Indenta o primeiro parágrafo de cada seção.
\usepackage{color}				% Controle das cores
\usepackage{graphicx}			% Inclusão de gráficos
\usepackage{microtype} 			% para melhorias de justificação
% ---

% ---
% Pacotes de matemática
% ---

\usepackage{amsmath}
\usepackage{amsfonts}
\numberwithin{equation}{section}		
% ---

% ---
% Pacotes de citações
% ---
\usepackage[brazilian,hyperpageref]{backref}	 % Paginas com as citações na bibl
\usepackage[alf,abnt-etal-list=0,abnt-etal-cite=1]{abntex2cite}	% Citações padrão ABNT

% ---
% Configurações de estilos de capítulos
% ---

\renewcommand{\ABNTEXpartfontsize}{\normalsize}
\renewcommand{\ABNTEXchapterfontsize}{\normalsize}
\renewcommand{\ABNTEXsectionfontsize}{\normalsize}
\renewcommand{\ABNTEXsubsectionfontsize}{\normalsize}
\numberwithin{figure}{section}
\numberwithin{table}{section}
\linespread{0.95}
% ---
%\renewcommand*{\familydefault}{\sfdefault}
\renewcommand{\rmdefault}{phv}
\renewcommand{\sfdefault}{phv}
\renewcommand{\ttdefault}{pcr}
% ---

% ---
% Configuração da capa
% ---

\renewcommand{\imprimircapa}{%
	\begin{capa}%
		\center
		\ABNTEXchapterfont\ABNTEXchapterfontsize
		\imprimirinstituicao
		\vfill
		\imprimirtitulo
		\par
		por
		\par
		\imprimirautor
		\vfill
		RELATÓRIO PARCIAL
		\vfill
		\imprimirlocal, \imprimirdata
		\vspace*{1cm}
	\end{capa}
}

% ---

%%% -----
%%% Formato de cabeçalho/rodapé romano nos elementos pré-textuais
%%% -----

%% Novo estilo
\makepagestyle{estilo_pretextual} %%% escolha um nome
  \makeevenfoot{estilo_pretextual}{}{\ABNTEXfontereduzida \normalsize \thepage}{}
  \makeoddfoot{estilo_pretextual}{}{\ABNTEXfontereduzida \normalsize \thepage}{}

%% Customiza comando \pretextual
\renewcommand{\pretextual}{
  \pagestyle{plain}
  \pagenumbering{roman} %%% ou \pagenumbering{Roman}
  \aliaspagestyle{chapter}{estilo_pretextual}% customizing chapter pagestyle
  \pagestyle{estilo_pretextual}
  \aliaspagestyle{cleared}{empty}
  \aliaspagestyle{part}{estilo_pretextual}
}

\renewenvironment{resumo}[1][ ]{%
   \PRIVATEbookmarkthis{#1}
   \renewcommand{\abstractnamefont}{\chaptitlefont}
   \renewcommand{\abstractname}{\ABNTEXchapterupperifneeded{#1}}
   \begin{abstract}
  }{\end{abstract}\PRIVATEclearpageifneeded}

%%% -----
%%% Formato de paginação para elementos textuais
%%% -----
\makepagestyle{estilo_textual}
  \makeevenhead{estilo_textual}{}{}{\normalsize \thepage}
  \makeoddhead{estilo_textual}{}{}{\normalsize \thepage}

\renewcommand{\textual}{
	\pagestyle{plain}
	\pagenumbering{arabic}
	\setcounter{page}{1}
	\aliaspagestyle{chapter}{estilo_textual}
	\pagestyle{estilo_textual}
	\aliaspagestyle{cleared}{empty}
	\aliaspagestyle{part}{estilo_textual}
}

%%% -----
%%% Sem página de divisão para anexos e apendices
%%% -----

\renewcommand{\partapendices}{%
}
\renewcommand{\partanexos}{%
}

%%% -----

\addto\captionsbrazil{%
	\renewcommand{\bibname}{REFER\^ENCIAS BIBLIOGR\'AFICAS}
}

% --- 
% CONFIGURAÇÕES DE PACOTES
% --- 

% ---
% Configurações do pacote backref
% Usado sem a opção hyperpageref de backref
\renewcommand{\backrefpagesname}{Citado na(s) página(s):~}
% Texto padrão antes do número das páginas
\renewcommand{\backref}{}
% Define os textos da citação
\renewcommand*{\backrefalt}[4]{
	\ifcase #1 %
		Nenhuma citação no texto.%
	\or
		Citado na página #2.%
	\else
		Citado #1 vezes nas páginas #2.%
	\fi}%
% ---

% ---
% Informações de dados para CAPA
% ---
\titulo{MODELO DE RELATÓRIO PARCIAL COM \LaTeXe ~E ABN\TeX}
\autor{Ricardo Frederico Leuck Filho}
\local{Porto Alegre}
\data{março de 2015}
\instituicao{%
  MINISTÉRIO DA EDUCAÇÃO
  \par
  UNIVERSIDADE FEDERAL DO RIO GRANDE DO SUL
  \par
  DEPARTAMENTO DE ENGENHARIA MECÂNICA}
% ---

% ---
% Configurações de aparência do PDF final

% alterando o aspecto da cor azul
\definecolor{blue}{RGB}{0,0,50}

% informações do PDF
\makeatletter
\hypersetup{
     	%pagebackref=true,
		pdftitle={\@title}, 
		pdfauthor={\@author},
    	pdfsubject={\imprimirpreambulo},
	    pdfcreator={LaTeX with abnTeX2},
		pdfkeywords={abnt}{latex}{abntex}{abntex2}{trabalho acadêmico}, 
		colorlinks=true,       		% false: boxed links; true: colored links
    	linkcolor=blue,          	% color of internal links
    	citecolor=blue,        		% color of links to bibliography
    	filecolor=magenta,      		% color of file links
		urlcolor=blue,
		bookmarksdepth=4
}
\makeatother
% --- 

% --- 
% Espaçamentos entre linhas e parágrafos 
% --- 

% O tamanho do parágrafo é dado por:
\setlength{\parindent}{1.3cm}

% Controle do espaçamento entre um parágrafo e outro:
\setlength{\parskip}{0.2cm}  % tente também \onelineskip

% ---
% compila o indice
% ---
\makeindex
% ---

% ----
% Início do documento
% ----
\begin{document}

% Retira espaço extra obsoleto entre as frases.
\frenchspacing 

% ----------------------------------------------------------
% ELEMENTOS PRÉ-TEXTUAIS
% ----------------------------------------------------------
% \pretextual

% ---
% Capa
% ---
\imprimircapa
% ---

% ----------------------------------------------------------
% ELEMENTOS TEXTUAIS
% ----------------------------------------------------------
\textual

% ---
% Arquivo com o texto do relatório 
% ---
%%%%%%%%%%%%%%%%%%%%%%%%%%%%%%%
% TEXTO DO RELATORIO PARCIAL 
%%%%%%%%%%%%%%%%%%%%%%%%%%%%%%%

%%%%%%%%%%%%%%%%%%
%%%% OBJETIVO %%%%
\section{Objetivo}

%Implementar otimização topológica baseada em elasticidade não-local e
%comparar com métodos apresentados na literatura.
\begin{quote}Corresponde a responder a pergunta: O QUE tu farás como trabalho, ou ainda QUAL o tema do teu trabalho.
\end{quote}

O objetivo deste trabalho é se auto-justificar por meio da aplicação de lógica
sinuosa e convoluta, numa tentativa de convencer a banca do mérito do trabalho.

%%%%%%%%%%%%%%%%%%%%%%%
%%%% JUSTIFICATIVA %%%%
\section{Justificativa}

\begin{quote}
Deves expor aqui a relevância do tema escolhido. Corresponde a responder a
pergunta: POR QUE vais desenvolver esse trabalho?
\end{quote}

A importância de terminar o TCC e se formar é justificada pela diferença
entre o valor de uma bolsa de estágio e um salário de engenheiro.

\begin{equation}\label{eq:mat}
E_e = E_{min} + x_e^p(E_0 - E_{min}),~~ x \in [0,1]
\end{equation}

Um pouco de matemática sempre ajuda:

\begin{equation}\label{eq:obj}
	\begin{aligned}
		\underset{\mathbf{x}}{\text{minimizar}} :& \quad c(\mathbf{x}) =
			\mathbf{u}^T \mathbf{K u} =
			\sum_{e=1}^N E_e (x_e) \mathbf{u}_e^T \mathbf{K}_0 \mathbf{u}_e\\
		\text{sujeito a} :& \quad V(\mathbf{x})/V(\mathbf{1}) = f_v\\
				  & \quad\mathbf{K u} = \mathbf{f}\\
		    & \quad\mathbf{0} \leq \mathbf{x} \leq \mathbf{1}
	\end{aligned}
\end{equation}


E mais um pouco:

\begin{equation}\label{eq:sen}
\frac{\partial c(\mathbf{x})}{\partial x_e} = p x_e^{p-1}(E_0-E_{min})
\mathbf{u}^T_e \mathbf{K}_0 \mathbf{u}_e
\end{equation}

O problema descrito pelas equações \ref{eq:mat}, \ref{eq:obj} e \autoref{eq:sen}
pode ser visto em \citeonline[cap. 1.3]{bendsoetopology}.


%%%%%%%%%%%%%%%%%%%%%
%%%% METODOLOGIA %%%%
\section{Metodologia}

\begin{quote}
Deves mostrar aqui os passos que serão feitos para a realização do trabalho.
Devem ser citados os procedimentos, tais como revisão dos assuntos relacionados,
a montagem de algum projeto, o cálculo de alguma instalação, a comparação econômica, etc.
Corresponde a responder a pergunta: COMO vai ser feito o trabalho?
\end{quote}

O trabalho pode ser dividido nos seguintes passos:

\begin{itemize}
\item Revisão da literatura referente a otimização topológica;
\item Formulação de uma proposta de método;
\item Implementação do método proposto;
\item Aplicação do método a problemas padrão;
\item Avaliação da performance do método por comparação com resultados
	analíticos encontrados na literatura e com resultados de métodos existentes.
\end{itemize}

Os passos serão feitos em ordem ou não.

%%%%%%%%%%%%%%%%%%%%%%%
%%%% ESTÁGIO ATUAL %%%%
\section{Estágio atual do desenvolvimento}\label{cap_crono}

\begin{quote}
Indicar o estado atual do trabalho, seguido de um cronograma semanal. Esse cronograma deve
incluir todas as etapas do trabalho até a sua apresentação oral.
\end{quote}

Atualmente o trabalho está na etapa onde coisas precisam ser feitas (ver cronograma na \autoref{t:crono}).

\begin{table}[hb]
	\caption{\label{t:crono}Cronograma do trabalho.}
	\center
	\includegraphics[width=.98\textwidth]{crono.pdf}
\end{table}

% ---

% ----------------------------------------------------------
% Finaliza a parte no bookmark do PDF
% para que se inicie o bookmark na raiz
% e adiciona espaço de parte no Sumário
% ----------------------------------------------------------
\phantompart

% ----------------------------------------------------------
% Referências bibliográficas
% ----------------------------------------------------------
\bibliography{referencias}

\end{document}